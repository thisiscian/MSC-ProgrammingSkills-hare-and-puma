% A '%' character causes TeX to ignore all remaining text on the line,
% and is used for comments like this one.

\documentclass[12pt]{article}    % Specifies the document style.
\linespread{1.25}
\setlength\textwidth{7in}
\usepackage[margin=2.5cm, a4paper]{geometry}
\usepackage{palatino}
%\usepackage{dsfont}
\usepackage{amsthm}
\usepackage{amsmath}
\usepackage{enumerate}
\usepackage{amssymb}
\usepackage{graphicx}
\usepackage{caption}
\usepackage{subcaption}
\usepackage{array}
\usepackage{verbatim}
\usepackage{color}
\usepackage{graphics}
%\usepackage{hyperref}
\numberwithin{equation}{section}

\graphicspath{{../}}
%theoremstyle{plain}
%\newtheorem{example}{Example}[subsection]


\newcommand{\spand}{\hspace{5mm} \& \hspace{5mm}}
\newcommand{\rasp}{\hspace{5mm} \Rightarrow \hspace{5mm}}
\newcommand{\cf}[2]{\left(\frac{#1}{#2}\right)}
\newcommand{\pf}[2]{\frac{\partial{#1}}{\partial{#2}}}
\newcommand{\lm}[2]{\lim_{#1\rightarrow #2}}

\title{Threaded Programming, Coursework Part 1}  % Declares the document's title.
\author{a878-575d7e}    % Declares the author's name.


\date{}   % Deleting this command produces today's date.

\begin{document}           % End of preamble and beginning of text.
\pagenumbering{roman}

\maketitle                 % Produces the title.
\begin{center}

\abstract{}
\clearpage
\pagenumbering{roman}
\end{center}

\pagenumbering{arabic}
\section{Introduction}

\section{Tasks}
I was in charge of the update operation part of the program.
It was my job to ensure that the actual equations for updating the hares and pumas on the board were implemented correctly and efficiently.
I began by coding up the update equations in a seperate function, first simply in a way that it would work.
Following this I wrote unit tests in order to ensure that the operation was working correctly.
This included testing different inputs and ensuring that they behaved as expected.
For example starting with no pumas and ensuring no pumas appear when the update is applied.
Also checked was that the animals diffused from a starting point when there were no animals anywhere else on the grid.
It was also ensured that updates didn't occur on water sites in order to reduce the work necessary and ensuring that no animals were present there.
The original code did not meet some of these requirements due to mistakes and omissions so it was necessary to update it and ensure all the tests were passed.

The next thing I needed to worry about was the performance of the code.
Since the function I had written is called many times from the main program it was executed a lot and some profiling needed to be done.

\section{Results}

\section{Conclusions}


\end{document}

	\begin{figure}[ht]
		\centering
		\input{../loop1-comp}
		\caption{Performance of different scheduling against chunksize for loop 1}
		\label{Figure 1:}
	\end{figure}
