% A '%' character causes TeX to ignore all remaining text on the line,
% and is used for comments like this one.

\documentclass[12pt]{article}    % Specifies the document style.
\linespread{1.25}
\setlength\textwidth{7in}
\usepackage[margin=2.5cm, a4paper]{geometry}
\usepackage{palatino}
%\usepackage{dsfont}
\usepackage{amsthm}
\usepackage{amsmath}
\usepackage{enumerate}
\usepackage{amssymb}
\usepackage{graphicx}
\usepackage{caption}
\usepackage{subcaption}
\usepackage{array}
\usepackage{verbatim}
\usepackage{color}
\usepackage{graphics}
\usepackage[utf8]{inputenc}
%\usepackage{hyperref}
\numberwithin{equation}{section}

\graphicspath{{../}}
%theoremstyle{plain}
%\newtheorem{example}{Example}[subsection]


\newcommand{\spand}{\hspace{5mm} \& \hspace{5mm}}
\newcommand{\rasp}{\hspace{5mm} \Rightarrow \hspace{5mm}}
\newcommand{\cf}[2]{\left(\frac{#1}{#2}\right)}
\newcommand{\pf}[2]{\frac{\partial{#1}}{\partial{#2}}}
\newcommand{\lm}[2]{\lim_{#1}\rightarrow {#2}}
\newcommand{\pa}{P\'{a}draig \'{O}'Conbhu\'{i}}

\title{Programming Skills}  % Declares the document's title.
\author{Ruairi Short}    % Declares the author's name.


\date{}   % Deleting this command produces today's date.

\begin{document}           % End of preamble and beginning of text.
\pagenumbering{roman}

\maketitle                 % Produces the title.
\begin{center}

\abstract{}
\clearpage
\pagenumbering{roman}
\end{center}

\pagenumbering{arabic}
\section{Introduction}
The objective of this coursework was to design and implement a working population dynamics model in a group.
Hares and Pumas were the animals to be modelled using the equations
\[\pf{H}{t}=rH-aHP+k\left(\frac{\partial^2H}{x}+\frac{\partial^2H}{y}\right)\]
\[\pf{P}{t}=bHP-mP+l\left(\frac{\partial^2P}{x}+\frac{\partial^2P}{y}\right)\]
I was part of a group of 3 with Cian Booth and \pa.
We decided as a group how tasks should be split between the members and what each had to do.
It was decided that everyone should be in charge of their own unit testing.

\section{Tasks}
I was in charge of the computational kernal of the program.
It was my job to ensure that the actual equations for updating the hares and pumas on the board were implemented correctly and efficiently.
First the update equations were coded up in a seperate function, simply in a way that it would work.
Following this unit tests were written in order to ensure that the operation was working correctly.
This included testing different inputs and ensuring that they behaved as expected.
For example starting with no pumas and ensuring no pumas appear when the update is applied.
Also checked was that the animals diffused from a starting point when there were no animals anywhere else on the grid.
It was also ensured that updates didn't occur on water sites in order to reduce the work necessary and ensuring that no animals were present there.
The original code did not meet some of these requirements due to mistakes and omissions so it was necessary to update it and ensure all the tests were passed.\\

The next thing I needed to worry about was the performance of the code.
The program Valgrind was used in order to profile the code.
It was first seen that the update\_animals function was taking up a significant portion of the execution time.
To this end it was decided to attempt to optimise the code.
In order to perform tests on the function a performance test was written.
It was decided to use a relatively large board and a medium number of iterations so that the time was long enough to see performance.
It was also desired that the number of iterations would show up any unnecessary overheads that were incurred with calling the function.
The chosen size was a 1000x1000 board for 50 iterations.
Once this performance test was written, it was added to the build file for ease of use.
Optimisation could be done by using a number of techniques.
First the loop index ordering was checked.
Since Padraig had written a board class it was necessary to ensure that the index ordering was the same as in the C standard.
Due to the fact that it was and the loops had been written to account for this there was no optimisatino to be done.
Next the computation for which sites around the site to be updated was moved from inline with the computation of the update to computing it before hand.
Following this the computations of the numbers of hares and puma in adjacent squares were moved to be with the land computation in order to improve memory coherency.
This had a large impact on the performance since it meant that memory acceses were kept closer together bringing eecution time to 15 seconds.
In order to compute the update a new board was declared and the updates computed onto this board and then copied back to the main one for the main program to see.
It was noted that originally this new board was being declared everytime the function was called so the addition of a static clause meant that the board needed only be declared once.
This addition almost halved the execution time down to 9 seconds for the performance test.

\section{Learning Outcomes}
Throughout the course of this project I learned about a number of new things.
At the start of the project I felt that my programming skills may not have been as good as some of the others in my group but I feel that I have narrowed this gap as a result of this exercise.
Before this project I had never used version control.
As this was a necessary part of working together as a group I learned a lot about the use of git and the online repository github.
This knowledge will be very useful to me in the future as a programmer.

I also learned about the use of automake.
As part of the project it was necessary to have a build tool.
The group decided to used automake as this allowed for easy compilation of the code and meant that tests could be executed easily and thus were done more often,

Another aspect of this project that was new to me was the use of unit tests.
These showed themselves to be an integral part of designing and implementing an program.
Using unit testing I found a number of bugs that I would not have found otherwise.
I learned about implementing them and ensuring that a working code was always maintained so that all group members could work on their part without being hindered by what others had done.

I learned about using the program Valgrind in order to profile code and saw why it is important to profile code in order to see where it is most useful to put work into optimisation.

Finally I learned about working in a group on a coding project as this was something new to me.
I found that it was necessary to keep the written code to a high standard to ensure readability by all parties involved.
It was also important to be able to explain why something was done in a certain way since the other members were able to view the code at any time and discuss the method used.
This made me think more about what I was doing in general and resulted in an improvement in my performance on this project and my overall skills as a programmer.


\section{Group Dynamic}
Overall the group worked very well together. 
It was easy to come to an agreement on standards at the beginning and everyone was prompt with getting the work they had to do complete.
Also all members were always available to explain any bits of their code any others didn't understand immidiately and this meant that the group could work all the faster.
My rating of the group members performances are outlined below.

\begin{tabular}{|c|c|c|c|c|c|}
\hline
Name & Exceptional & Above average & OK & Below average & Poor\\
\hline\hline
Ruairi Short(Me) & & x& & &\\ 
\hline
Cian Booth & x& & & &\\
\hline
\pa & x& & & &\\
\hline
\end{tabular}
\\
I felt that Cian Booth and \pa did the tasks that were assigned to them to a high standard.
They did more than what was asked of them in most situations and this made the overall project better in the end.
They were also readily available to discuss the project and this made the entire experience much easier.
Everyone involved was extremely helpful throughout the entire period and all of the above make me feel that they performed exceptionally

I feel that I performed above average as I not only completed my tasks but also was available to fill in any extra work that needed to be done and ensured that all of my work was to a high standard and completely compatible with the rest of the program.

\end{document}

	\begin{figure}[ht]
		\centering
		\input{../loop1-comp}
		\caption{Performance of different scheduling against chunksize for loop 1}
		\label{Figure 1:}
	\end{figure}
