\section{Assignment of Tasks}
\vspace{-2em}\rule{\textwidth}{1pt}\vspace{1em}

Once a plan had been made it was necessary for the tasks associated with the program to be divided up among the group.
The first task was to decide on a build tool to use.
\texttt{Automake} was settled upon and P\'{a}draig took charge of implementing this as he had previous experience with its use.
Next a method of version control was decided upon.
\texttt{Git} was the chosen tool and an online repository was managed by Cian on the website \texttt{github}
Since main program naturally divided itself into three parts each team member was then polled and was assigned to the task they would most like to do.
The split was thus as follows;
\begin{itemize}
\item Input - \pa
\item Update Operation - Ruairi Short
\item Output - Cian Booth
\end{itemize}

P\'{a}draig did the input part of the program as he had the most experience with \texttt{C++} features such as memory allocation that was necessary to implement this part correctly and efficiently.

Ruairi was in charge of the update operation.
This part involved the most maths and having a background in physics meant that this part was ideal for him to work on.
It was also necessary to have tests that would ensure that the update was working correctly.
In order to this the equations needed to be solved analytically for certain situations and tests created to reproduce these situations.

Cian had a desire to implement the output part of the code and had previous experience with writing GUIs in \texttt{C++}.

Each group member was also responsible for testing their own part of the code.
This was since they would have the best understanding of the module they had written and could test it most effectively.
It also meant that it was possible to maintain a working version of the code throughout the entire project.

Once the code program was running to a sufficient standard the group met again to look at what further tasks could be done.
The conclusion was that the remaining tasks were attempting to improve the performance of the program overall, implementing a graphical user interface(gui) and implementing tests to check overall functionality of the main program.
It was decided that some time should be spent trying to improve performance.
It was speculated that the most likely user of CPU time would be the update function so Ruairi took charge of performance analysis and optimisation.
This involved profiling the code to find out if the speculation had been correct and acting on the data from the profiling.
Cian did the work on creating a gui using \texttt{ncurses} so that the output from the program could be better understood by the user as the PPM output was not immediately available while the program was running.
P\'{a}draig did further work testing the input part of the code to ensure that it was working efficiently and correctly.
