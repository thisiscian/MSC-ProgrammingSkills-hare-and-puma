\section{Further Work}
\vspace{-2em}\rule{\textwidth}{1pt}\vspace{-1em}
\subsection{Input}
  The program fails to build with the current framework when pgcpp is used
  as the \texttt{C++} compiler.
  The problem appears, on the surface at least, to be with the boost library.
  Future work might entail the removal of the boost program\_options library
  in favour of wither a custom option, involving parsing a rigid input file,
  or perhaps more simply, requiring variables to be set on compile.
  This may, in fact, not be the problem, as the compilation attempt with
  pgcpp was only done on the Morar machine.
  It may be possible to bundle the program\_options library with the
  source of this project, getting a user to compile it with pgcpp along
  with the rest of the code.
\subsection{Update}
\subsection{Output}
	There is a lot of work that could be done with the output section.
	The most pressing thing that needs to be done is testing the GUI.
	Although this is difficult, due to changing terminal size, it could be possible to fix this, as far as the program is concerned, using
	pre-processing conditionals in the code, forcing the ncurses set values of COLS and ROWS to standard values.
	
	Full user interactivity with the \texttt{write\_adjustable\_ppm} function could also be added, which could be useful for smaller input files.

	The ncurses section of the program was also not implemented very efficiently.
	It spends extended periods of time waiting, time which could be spent calculating upcoming iterations.
	Extraneous iterations could be saved in a file, and whilst this is potentially a way to use up large amounts of memory, it is worth looking into.
	
	Finally, if a more universally accessible, and less cumbersome windows based GUI could be found, the quality and usability of the program could be greatly enhanced, as the majority of people are very used to using windows based programs.
