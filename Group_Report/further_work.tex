\section{Further Work}
\vspace{-2em}\rule{\textwidth}{1pt}\vspace{-1em}
\subsection{Input}
  The program fails to build with the current framework when pgcpp is used
  as the \texttt{C++} compiler.
  The problem appears, on the surface at least, to be with the boost library.
  Future work might entail the removal of the boost program\_options library
  in favour of wither a custom option, involving parsing a rigid input file,
  or perhaps more simply, requiring variables to be set on compile.
  This may, in fact, not be the problem, as the compilation attempt with
  pgcpp was only done on the Morar machine.
  It may be possible to bundle the program\_options library with the
  source of this project, getting a user to compile it with pgcpp along
  with the rest of the code.
\subsection{Update}
  In the update section there is more work to be done in optimising the code to run faster.
  However one can almost always say this about a piece of code!
  A number of possibilities were considered but due to time constraints were not implemented.

  For example to reduce further the overhead of copying the board back from the working board back to the board visible from the main program one could simply move the pointer.
  This could be quite difficult to implement in reality as it is not known exactly how this might behave with the board class but it could reduce the number of operations and this increase performance.
  Alternatively having the working board initialised in the main program could potentially speed up execution.

  There is also more work to do from the point of view of parallelizing the loop.
  Different loop scheduling options for OpenMP could be experimented with in order to find an optimum.
  The use of a message passing protocol such as MPI could also be tested as the communications pattern is quite regular and could possibly see good performance.

  It is possible that there is more to be done with improving cache accesses and overheads associated with loops by techniques such as loop unrolling as well but it was decided to not implement any of these in order to keep the code readable.
\subsection{Output}
	There is a lot of work that could be done with the output section.
	The most pressing thing that needs to be done is testing the GUI.
	Although this is difficult, due to changing terminal size, it could be possible to fix this, as far as the program is concerned, using
	pre-processing conditionals in the code, forcing the ncurses set values of COLS and ROWS to standard values.
	
	Full user interactivity with the \texttt{write\_adjustable\_ppm} function could also be added, which could be useful for smaller input files.

	The ncurses section of the program was also not implemented very efficiently.
	It spends extended periods of time waiting, time which could be spent calculating upcoming iterations.
	Extraneous iterations could be saved in a file, and whilst this is potentially a way to use up large amounts of memory, it is worth looking into.
	
	Finally, if a more universally accessible, and less cumbersome windows based GUI could be found, the quality and usability of the program could be greatly enhanced, as the majority of people are very used to using windows based programs.
