\section{Conclusion}
\vspace{-2em}\rule{\textwidth}{1pt}\vspace{1em}

A program that could visualise the population dynamics of hares and pumas was developed.
It included two output modes, writing .ppm files, and showing the output with an ncurses GUI.
The input was implemented, allowing users to include their own input files, and their own parameters.

The group worked together very well, which was especially down to the good planning done before the actual coding was done.
The separation of work went well, as we all finished up roughly within a day of each other.
The work was given to the appropriate people, as befitted their egregious talents.

The program was produced using many Programming Skills techniques.
A combination of git and github was used for revision control.
Automake was used as a build tool.
Valgrind was used as a profiling suite.
The tests implemented saved a lot of time and bug checking.
Debugging was done by hand, as the testing and program design meant that big bugs didn't really appear.
The code was optimised using compiler flags, and parallelised using OpenMP.

In conclusion, the project was a success.
