\section{Build Tools}
\vspace{-2em}\rule{\textwidth}{1pt}\vspace{1em}

For this project, the \texttt{Autotools} suite was chosen.
This is a free set of tools for generating portable programs for many
Unix-like systems.

\subsection{Other Tools}
The other main competitor for use in this project was plain \texttt{make}.
While a plain \texttt{Makefile} would be easier for a small, simple project,
we felt that \texttt{Automake} was a more robust build system, better suited
to a project such as this one.
Two others that came up were \texttt{cmake}, when another group mentioned it
was their build tool of choice, and \texttt{qmake}, when \texttt{Qt} was
considered for use in building a GUI.
However, much time and effort had already been invested in setting up
an \texttt{Automake} build environment, and the other build systems offered
no particular advantages worth switching environments for.

\subsection{Features}
The \texttt{Autotools} suite is a widely used set of tools for generating
portable makefiles and configuration scripts.
Since the tools are widely used, many users are already familiar with
the procedure for compiling a project using these tools.
This is the familiar ``\texttt{./configure \&\& make \&\& make install}''
combination.
The configuration script can be used to find or explicitly specify the
locations of certain libraries on a system, along with specifying install
paths for the compiled program.
It as also used to ensure the appropriate build tools, libraries, and
library headers exist on a system, and give some descriptive error
messages when these can't be found.
The alternative is for a user to attempt to decrypt some mysterious
error messages when a program either outputs errors at compile time, or does
compile and outputs an error at run time.

The \texttt{Automake} tool specifies a simple way of defining programs to be
compiled, the dependencies of these programs, and the flags to be passed
to them.
The tool makes it simple to specify these things on a global level, or a
per program level.
It also define some standard \texttt{Makefile} targets, such as
\texttt{clean}, \texttt{distclean}, \texttt{check} and \texttt{dist}
which have some standard functions.
These are supplied with no extra configuration, whereas with a plain
\texttt{Makefile}, these would have to be hand-coded.
\texttt{Automake} also reduces the need for ``clever'' \texttt{Makefiles},
which are almost always a nightmare for anyone other than the original
author to maintain (assuming that they themselves understand it).

Also provided is a rudimentary test environment.
Programs that might be compiled for a test run are added to a list,
separate from the ones that are compiled for installation,
and the programs to be run during a test run are added to another list.
For this project, it turned out that just one test environment
wasn't quite enough, but since \texttt{Automake} provides
the means for including pure \texttt{make} targets, it was trivial
to add new targets for compiling and running specific sets of tests.
This meant the main tests could be run by simply running
``make check'', the performance tests coule be run by running
``make performancetest'' and the gui tests could be run by running
``make guitest''.
This makes testing slightly less painful for the programmer.

The main problem with the \texttt{Autotools} suite is that is can have
quite a steep learning curve to learn all the gory details involved
in it.
However, in most cases, these can simply be ignored.
This is because of one of the main strengths of the suite: its widespread
use.
That is, there are plenty of tutorials for using the tools and plenty
of questions and answers available on the internet.
Another problem is that the configuration script can take a while to
run, and needs to be run again if any changes are made to any of the
files used to generate the \texttt{configure} script or any of the
\texttt{Makefiles}.
This can be quite tedious.
