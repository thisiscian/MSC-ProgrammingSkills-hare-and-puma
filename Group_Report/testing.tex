\section{Testing}
Three sets of tests were set up for this project:
\begin{itemize}
  \item Correctness tests
  \item Performance tests
  \item GUI tests
\end{itemize}

\subsection{Correctness Tests}
The approach taken for correctness testing was to write one program to
do each unit test.
Upon the first error in the test, the program would exit early and return
a value of 1.
The idea behind this is that it would eliminate the overhead of learning
a whole unit testing framework that might return the number of tests passed
and the number failed.
We felt this would be unnecessary, since if only one test fails, it's not
really necessary to see how many others passed and how many failed, as the
program is broken, regardless of the outcome of other tests.
If more details are required, a programmer can simply comment out the offending
test to see which others pass and fail.

Finally, an integration test was built in the form of a bash script that
passes several input files into the program and ensures it returns the
correct output in some special cases.

\begin{itemize}
  \item Given the equations
    \begin{align*}
      \frac{\partial{H}}{\partial{t}} &= -aHP \\
      \frac{\partial{P}}{\partial{t}} &=  bHP
    \end{align*}
    The following equality holds
    \begin{equation*}
      \frac{\partial{H}}{\partial{t}}
        = -\frac{a}{b}\ \frac{\partial{P}}{\partial{t}}
    \end{equation*}
    This should test that the two PDEs are properly coupled.

  \item Given an input with the hare or puma densities in some squares
        set to zero, when the diffusion terms are set to zero, these
        densities should remain at zero.

  \item When only the hare birth term, $r$, and puma death term, $m$, are
        nonzero, the hare and puma populations should follow the equations
        \begin{align*}
          H(t) &= H_0 e^{rt} \\
          P(t) &= P_0 e^{-mt}
        \end{align*}

  \item When only diffusion terms are nonzero, both hare densities and pume
        densities should reach an equilibrium distribution.

  \item When an equilibrium distribution is given as input to the program,
        given the same input values as the previous test, the program should
        output the exact same board (or reasonably close to it, as.

  \item When a checkerboard distribution is given and only diffusion terms
        are nonzero, zero density tiles should not stay at zero.
\end{itemize}


\subsection{Performance Tests}
\subsection{GUI Tests}
