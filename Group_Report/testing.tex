\section{Testing}
Three sets of tests were set up for this project:
\begin{itemize}
  \item Correctness tests
  \item Performance tests
  \item Output tests
\end{itemize}

\subsection{Correctness Tests}
The approach taken for correctness testing was to write one program to
do each unit test.
Upon the first error in the test, the program would exit early and return
a value of 1.
The idea behind this is that it would eliminate the overhead of learning
a whole unit testing framework that might return the number of tests passed
and the number failed.
We felt this would be unnecessary, since if only one test fails, it's not
really necessary to see how many others passed and how many failed, as the
program is broken, regardless of the outcome of other tests.
If more details are required, a programmer can simply comment out the offending
test to see which others pass and fail.

Finally, an integration test was built in the form of a bash script that
passes several input files into the program and ensures it returns the
correct output in some special cases.

\begin{itemize}
  \item Given the equations
    \begin{align*}
      \frac{\partial{H}}{\partial{t}} &= -aHP \\
      \frac{\partial{P}}{\partial{t}} &=  bHP
    \end{align*}
    The following equality holds
    \begin{equation*}
      \frac{\partial{H}}{\partial{t}}
        = -\frac{a}{b}\ \frac{\partial{P}}{\partial{t}}
    \end{equation*}
    This should test that the two PDEs are properly coupled.

  \item Given an input with the hare or puma densities in some squares
        set to zero, when the diffusion terms are set to zero, these
        densities should remain at zero.

  \item When only the hare birth term, $r$, and puma death term, $m$, are
        nonzero, the hare and puma populations should follow the equations
        \begin{align*}
          H(t) &= H_0 e^{rt} \\
          P(t) &= P_0 e^{-mt}
        \end{align*}

  \item When only diffusion terms are nonzero, both hare densities and pume
        densities should reach an equilibrium distribution.

  \item When an equilibrium distribution is given as input to the program,
        given the same input values as the previous test, the program should
        output the exact same board (or reasonably close to it, as.

  \item When a checkerboard distribution is given and only diffusion terms
        are nonzero, zero density tiles should not stay at zero.
\end{itemize}


\subsection{Performance Tests}
\subsection{Output Tests}
	Output testing for writing \texttt{.ppm}s was done in a comprehensive fashion.
	It is simple to test output files, but inputting a known file, and immediately outputting that file.
	If the values in the output file correspond to the input, then the system works.

	Several cases were developed to test the output;
	\begin{description}
		\item[First Line Validity]
			A plain PPM file will be invalid if the first line does not contain the correct information.
			It requires the text "P3", to indicate that it is a plain PPM.
			It then needs the width and height of the data to be drawn.
			Finally, it needs to know the maximum value of the data within (which is normally 255).
			This test ensures that a sample input file of size $5\times5$ creates an output file with the first line reading "P3 5 5 255"
		\item[Number of Lines]
			Plain PPM files also require that the number of lines in the file is correct. 
			The number of lines in the document should correspond to the second digit indicated on the first line plus 1 (to include the first line).
			Again, an input of size $5\times5$ was used to check this.

			Note that the output of this program actually introduces extra spaces between lines, to increase ASCII readability of the file.
		\item[Number of Elements]
			A final check of the validity of the plain PPM output file, that the number of pixels represented is the same as the number of input elements.
			The program outputs the data as three integers, with spaces separating them.
			Between each data point, is a tab space.
			This can be used to count the number of pixels drawn in each line, which is the number of elements in the output grid.
			In combination with the above test, this implies that the output file is indeed laid out correctly.
		\item[No Hares, No Pumas, Only Land]
			In this test, the system is set into a state where there are no rabbits, hares, and there is only land.
			In this case, all the pixel values in the output file should be "0 0 0".
		\item[No Hares, No Pumas, Only Water]
			In this test, the system is set as above, except with the entire region being made of water.
			If this is the case, the pixel values should all be "0 0 255", indicating no animals and only water.
		\item[Constant Hares, Constant Pumas]
			In this test, as the above test, except there are now non-zero numbers of hares and pumas.
			Again, the expected result is that the pixel values are "0 0 255", ignoring the data of the hare and puma.

			This test should have no particular use, as water tiles cannot, in the program, be populated by any animals.
			However, it does ensure that the water system is printing correctly.
		\item[Constant Hares, No Pumas, Only Land]
			Here, the number of hares is the same in all grid points, 1.
			The region is entirely made of land.
			Therefore the pixel count of each element should be "0 1 0".
			
			This test ensures that hares are being added to the output properly.
		\item[No Hares, Constant Pumas, Only Land]
			As above, but with no hares and every puma set to a value of 2.
			The pixel count is expected to be "2 0 0" for each element.

			This test ensures that pumas are being added to the output properly.
		\item[Constant Hares, Constant Pumas, Only Land]
			Now there is a constant number of hares, 1, and a constant number of pumas, 2, in each element.
			This leads to a pixel count of "2 1 0".
			
			This test makes sure that the hares and pumas are not somehow interferring with each other's output.
		\item[Linear Hares, No Pumas, Only Land]
			Now each element has a number of hares equal to the sum of it's colum and row minus 1, if they are numbered from 1 to 5.
			Thus, the first element in the first row is expected to have a pixel value of "1 0 0", the second "2 0 0" etc., all the way to the last element on the last row having a pixel value of "9 0 0".

			This test ensures that the number of being outputted is independent of other values of hare.
		\item[No Hares, Linear Pumas, Only Land]
			As above, but with pumas, and the rows and colums are numbered from 5 to 1.
		\item[Non-Integer Hares, No Pumas, Only Land]
			This test is as the Linear Hares, No Pumas, Only Land test, except that each element is now multiplied by 1.1.
			This means that a full range of decimal places is tested.
			It is expected the output will round the value down, giving the same results as the aforementioned test.

			This test ensures that the plain PPM does not contain any non-integer values.
		\item[No Hares, Non-Integer Pumas, Only Land]
			As above but for pumas.
		\item[Sample Input]
			The sample input is a much larger file type, $24\times24$ elements.
			It  contains regions with water, regions with no hares or pumas,with just hares, with just pumas, and regions with both species.
			This stress tests the system to make sure that the previous tests pass in unison.
		\item[Sample Input Xstretch]
			This is the same as the above system, except that now the board is rectangular, having been stretched by one in the x direction to $25\times24$.

			This ensures that the output is not reliant on symmetric systems.
		\item[Sample Input Ystretch]
			As above, except that it has been stretched in the y direction, to become $24\times25$.
			
			This test ensures that there is not some special dependence on either symmetry of the board or the x direction being stretched.
	\end{description}
