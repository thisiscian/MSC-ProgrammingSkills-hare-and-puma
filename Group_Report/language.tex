\section{Programming Language}
\vspace{-2em}\rule{\textwidth}{1pt}\vspace{1em}

As a group, it was decided that the programming language \texttt{C++} was the right language to use for this project.
There were several reasons for this.
The main reason was that each member of the group was well versed in the use of \texttt{C++}.
Familiarity ensures that the production of code is fast and efficient.
Furthermore, using a programming language that the group was accustomed to, meant that pertinent libraries could be used.

Another important reason to use \texttt{C++} is that, compared to the likes of \texttt{Fortran}, input and output operations are considerably easier, and better implemented.
In a project where reading and writing files is a requirement, it is obviously beneficial to use a programming language that is strong in this area.

Equally, \texttt{C++} has many libraries (courtesy of \texttt{C} compatibility) and compilers that ensure optimised code, that can run at the highest efficiency, unlike languages like \texttt{Java}.
In particular, the library \texttt{OpenMP} is particularly useful for improving efficiency.

\texttt{C++} was chosen over \texttt{C} because of the inherent memory management within.
This is useful in a project which could potentially be using very large arrays.

Finally, again courtesy of the \texttt{C} compatibility, \texttt{C++} has access to many different varieties of GUI libraries.
These include, but are not limited to, the very commonly used \texttt{wxWidgets} (seen in programs like Audacity and Dropbox), the popular library \texttt{Qt} (used by programs like Skype and Mathematica) and the terminal based \texttt{ncurses}, used by the likes of Screen and cmus.
Thus it is a good choice.
